\chapter{Introdução}
\label{sec-intro}
\vspace{-1cm}

Este documento apresenta o projeto (\textit{design}) do sistema \emph{SAGUP - Sistema de Administração e Gestão de Usuários e Permissões}. O SAGUP tem como principal objetivo oferecer uma plataforma segura, eficiente e acessível para o gerenciamento centralizado de usuários, seus respectivos perfis (roles) e permissões de acesso em sistemas corporativos.

O sistema permitirá realizar autenticação de usuários, cadastro e edição de perfis, atribuição de permissões específicas, bem como a visualização e auditoria de acessos. Essas funcionalidades são fundamentais para organizações que precisam garantir o controle rigoroso sobre quem pode acessar quais recursos, de forma auditável e escalável.

O SAGUP visa resolver lacunas frequentemente encontradas em sistemas legados ou soluções improvisadas, promovendo maior governança digital e conformidade com boas práticas de segurança da informação.

Além desta introdução, este documento está organizado da seguinte forma: 
a Seção~\ref{sec-plataforma} apresenta a plataforma de software utilizada na implementação do sistema;
a Seção~\ref{sec-rnfs} apresenta a especificação dos requisitos não funcionais (atributos de qualidade), definindo as táticas e o tratamento a serem dados aos atributos de qualidade considerados condutores da arquitetura; 
a Seção~\ref{sec-arquitetura} apresenta a arquitetura de software; por fim, 
a Seção~\ref{sec-frameweb} apresenta os modelos FrameWeb que descrevem os componentes da aplicação, com base na metodologia empregada.
